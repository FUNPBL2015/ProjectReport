% 以下のコメントはよく読んだほうが良いと思います。
\documentclass[twocolumn,draft]{jsarticle}
 
 %%%%%%%%%%%%%%%%%%%%%%%%%%%%%%%
 \usepackage{url}
 %%%%%%%%%%%%%%%%%%%%%%%%%%%%
 
 
\usepackage{graphicx,color}%図版を取り込む場合は必要。
\usepackage{okumacro}% 振り仮名を使用するときは必要。
%
\ifdraft
 % \def\hissu{\bgroup\color{red}}
   \def\hissu{\bgroup\color{black}}
  \def\endhissu{\egroup}
\else
  \def\hissu{}
  \def\endhissu{}
\fi
% % % % % % % % % % % % % % % % % % % % % % % % % % % % % %
% ここから↓
\pagestyle{empty}
\advance\textheight\headheight \headheight=0pt
\advance\textheight\headsep    \headsep=0pt
\advance\textheight\footskip   \footskip=0pt
\textheight=738truept
\advance\textwidth\marginparsep \marginparsep=0pt
\advance\textwidth\marginparwidth \marginparwidth=0pt
\advance\textwidth\oddsidemargin \oddsidemargin=0pt
\evensidemargin=\oddsidemargin
\textwidth=50zw
\advance\textwidth2zw
\columnsep=2zw
\topmargin=-5.4mm
\oddsidemargin=-7.4mm
\begin{document}
\fontsize{10}{18}\selectfont
% ↑ここまで変更しないほうが良いです。
% % % % % % % % % % % % % % % % % % % % % % % % % % % % % %

% ↓ここからが変更すべき箇所です。
\twocolumn[%
% 中間か最終かを記載する
\noindent{プロジェクト報告書(最終)Personal Middle Report}
% 個々に提出日を記入する
   \hfill{提出日 (Date) 2016/1/20}\par\vskip2zw
\begin{center}
%日本語のプロジェクト名
 {\LARGE フィールドから創る地域・社会のためのスウィフトなアプリ開発}\par\vskip1zw
%英語のプロジェクト名
 {\LARGE "Swift" Application Development Based on Field Research}\par\vskip1.5zw

%学籍番号、日英の名前
 {\large 1013220\quad 新保遥平\quad Yohei Shinpo}\par\vskip2zw
\end{center}]%


\section{プロジェクトの概要}
\begin{hissu}

本プロジェクトでは、現場(フィールド)を実際に調査し、そこにある問題を
解決するためのアプリケーション(以下アプリとする)を開発してきた。このアプリが地域や社会に
役立つことが目的である。このプロジェクトでは観光系・医療系・教育系に
3つのフィールドを設定し、グループに分かれて活動を行ってきた。

\end{hissu}

\section{観光系}
\subsection{背景}
\begin{hissu}
観光系グループは2016年、春の北海道新幹線開業に伴い、観光産業に力を入れ始めている
木古内町をフィールドに設定した。木古内町は、2016年春に開業する北海道新幹線の停車駅ができる。
それによって観光客の増加が見込まれることから、今現在、木古内町観光の基盤づくりや
観光地としての知名度向上活動に特に力を入れており、駅前道路の整備、
木古内町マスコットキャラクターのキーコがイべントでPRを行うなどの活動をしている。
\end{hissu}
\subsubsection{フィールドワーク}
\begin{hissu}
5月初旬に実際に木古内町に現地調査へ行き、どのような問題点があるのか、どのような魅力が
あるのかを調査した。木古内町は観光産業に力を入れているという現状がある一方で、
木古内町観光協会のWebサイトやFacebook、パンフレットなど様々なメディアで情報が発信
されているがそれらの情報が観光客や木古内町を知りたいと思っている人に効果的に伝わって
いないという問題があった。
\end{hissu}
\subsection{課題の設定と到達目標}
\begin{hissu}
アプリ開発を行うにあたり、大きく分けて2つの課題の解決に取り組む。まず1つ目は、観光情報の
アクセス性改善と魅力の発信である。現在問題となっている、観光情報が各種メディアに
分散しその各所がそれぞれに違った情報を提供しているため効果的に伝えられていないことを解決し、
木古内町観光を魅力的に伝える。
2つ目は観光した際に撮影した写真を振り返る機会が少ないことである。そこで写真を振り返る機会を
増やすことで再度、木古内町へ観光に行くリピーターの増加や、写真を通した思い出話から
広がる口コミによって木古内町の認知度が高まることに繋がると考えた。本プロジェクトでは
上記2点の問題を我々のアプリで解決することによって木古内町観光の満足度向上を図ることを
目標とした。
\end{hissu}
\subsection{課題解決のプロセスとその結果}
\subsubsection{開発の進め方}
\begin{hissu}
観光系グループではPDCAサイクル、スクラムなどのアジャイルソフトウェア開発の手法を
学び、実践しながらアプリを開発した。また、第1サイクルから第4サイクルまでの4度の開発を
行った。
\end{hissu}
\subsubsection{第1開発}
\begin{hissu}
第1サイクルはプロジェクト発足から中間発表会用のポスターレビューとした。
本サイクルでは木古内町の観光情報が分散していること、観光中に撮影した写真が整理されて
いないことを問題として定義した。そこで幾つかのアプリの機能案を考え、
6月12日に外部講師の方からレビュー受けた結果、現在の機能案に木古内らしさを
加える必要があるとの指摘を受けた。そこで指摘された点を修正し、実装した機能が以下の3つである。
\begin{itemize}
\item マップ上に表示されたピンをタップすると吹き出しとして詳細情報をクイック表示する機能
\item 木古内町の天気を表示する
\item 現在地から目的地までのルート案内
\end{itemize}
\end{hissu}
\subsubsection{第2開発}
\begin{hissu}
第2サイクルは中間発表会用のポスターレビューから中間発表会までとした。
本サイクルでは2つの問題に同時に取り組んでいった。
1つ目は第1サイクルで実装していた部分の情報の見せ方を改善することである。
2つ目は観光中に撮影した写真が整理されていないという問題への解決案として
フォトストーリーという機能を考案した。これはユーザが撮影した写真と移動した経路をマップ
で表示し、木古内町観光をよりリアルに振り返れる仕組みである。
\end{hissu}
\subsubsection{第3開発}
\begin{hissu}
第3サイクルは後期が始まった9月25日から10月21日に行われた外部講師による
リモートレビューまでとした。本サイクルでは、第2サイクルで見つかった大きな課題の一つで
ある写真を撮る行為にもっとワクワクするような付加価値を加えなければならないこと、
観光情報をより魅力的に紹介することを課題として活動を進めた。
まず、前期を振り返り、もう一度課題を見直した結果、撮った写真を利用してカルタという
「もの」にする機能を作ることも決定した。しかし、カルタは大量に紙を印刷するので手間が
かかるという課題が残った。本サイクルでは、アプリのタイトルを「キーコ紀行」と呼ぶことを正式に
決定した。
\end{hissu}


\subsubsection{第4開発}
\begin{hissu}
第4サイクルは外部講師のリモートレビューが終わった日から12月11日のプロジェクト最終報告会まで
とした。本サイクルでは、カルタに代わる何か手間のかからない「もの」を提案することを課題とした。
その結果リーフレットという案が生まれた。カルタのような数の手間をかけずに作れるので、
この案を採用した。リーフレットの表側には木古内の観光情報を記し、裏側にはユーザが
撮影した写真で自動的に割り当てあれるようにデザイン・実装をすることにした。
また、アプリ内でも振り返られるようにアルバム機能を作ることを決定した。本サイクルで
実装した機能は大きく分けて、観光する機能、振り返る機能、そして印刷する機能の3つである。
11月14日のアカデミックリンクでは、多くの企業の方や一般の方からレビューを受けた。
そのレビューを受けて最終報告会までさらに開発を続けた。本サイクルでは長く続けてきた
要件定義が充分に固まったため、比較的実装に集中できたため完成度の高いアプリの開発が出来た。
今後はアプリのリリースをすることになった。
\end{hissu}
\subsection{今後の課題と展望}
\begin{hissu}
最終成果発表会で得られたレビューをもとに第4サイクルについて改善を行った後、2月頃までの
リリースを目標としている。この時期にリリース目標を設定した理由としては、3月に北海道新幹線が
開業することが挙げられる。観光客が最も訪れると予想される開業直後に間に合うようリリース準備を
進めていく予定だ。今後の作業としては、まずレビューをもとに機能の拡張やUIの再設計を行い、
並行して既知のバグの修正を行っていく。これらが完了次第、App Storeへのリリースを行う。
また、リリース後にも定期的なメンテナンスを行っていくほか、季節ごとのコンテンツの追加などの
案も検討している。


\end{hissu}















\section{医療系}

\subsection{背景}
\begin{hissu}
近年、日本では高齢化が急速に進み、高齢者が占める人口の割合は超高齢社会と呼ばれる水準に
まで達した。それに伴い、日本の認知症患者数は増加していく傾向にありながら、認知症に関する知識を
提供する場が不足しているのが現状である。また高齢者については、会話の頻度が少なくなって
しまうことから、認知機能の低下に伴う認知症を発症するケースが多くみられる。また、認知症患者は
認知機能が低下しているため、家族や医師との意思疎通を上手く行うことができない。その結果、
自分の要望に合った納得のいく医療行為を受けることが難しくあるのも、現在の認知症医療における
問題点である。

\end{hissu}

\subsection{課題の設定と到達目標}
\begin{hissu}
本グループでは「多くの人が抱えている認知症に対する不安を解消して利用者の行動を支援すること」、「高齢者と離れて暮らしている家族が高齢者と共に楽しみながら認知症の対策を行えるようにすること」の2つを目的として活動を進めていく。この目的を達成するために現在、京都府立医科大学の成本医師らによって作成されている「意思決定支援マニュアル」と呼ばれる冊子を参考にした。
また上記の2つの目的を達成するために満たされるべき条件は以下の通りである。
\begin{itemize}
\item 認知症に関する正しい知識を提供し、認知症に対する不安を解消すること
\item 本人の医療行為に対する意思を確認する機会を提供することで、認知症発症後でも本人の
意思が反映された納得のいく医療行為を受けることに繋げること
\item 離れて暮らす高齢者とその家族がコミュニケーションをとる機会を提供し、お互いの状況を
把握し合えること
\item 認知症対策に繋がる情報を提供し、離れて暮らす高齢者とその家族が共に楽しみながら
認知症の対策を行えるようにすること
\end{itemize}
これらの条件を満たすために、マニュアルのアプリ化や認知症対策を行うことに繋がる機能の
実装を行っていく。
\end{hissu}

\subsection{課題解決のプロセスとその結果}
\subsubsection{第一開発}
\begin{hissu}
第一開発では紙媒体の「意思決定支援マニュアル」の一般市民・認知症高齢者家族向け
のアプリ化を行う。紙媒体のマニュアルをアプリ化する理由は、2つある。
1つ目は、一般市民・認知症高齢者家族など利用者によって変化する読みやすさや見やすさに対応することができると考えたからだ。また情報を文字で表示するだけでなく、音声ガイド機能や画面の拡大縮小の機能などを付ける。
2つ目は、データの記録や保存を、端末上で行うことにより、記録管理が容易になると考えたからだ。また、記入したデータを分析するなどの機能に繋げることができると考える。
開発の際は、より分かりやすく情報を伝えるため、マニュアルの文章を改変したものを用いる。

\end{hissu}

\subsubsection{第二開発}
\begin{hissu}
第二開発では、「コミュニケーションの中で認知症に関する正しい知識を得てもらい、認知症対策に
繋がる行動をしてもらうこと」を目的に活動を行った。中間発表で頂いた評価からマニュアルの
機能だけではなく、独自の機能を考えることにした。メンバー全員で独自の機能について考え、
認知症対策に繋がる写真を用いたコミュニケーション機能を追加することにした。マニュアル機能と
コミュニケーション機能を一つのアプリとして組み合わせるために、開発の計画や設計、実装を行った。
計画や設計を進め、プロトタイプが完成したため、JSTサイトビジットやアカデミックリンク2015に
参加し、医療関係者や一般市民の方からアプリのアイディアに関する評価を頂いた。また、
コミュニケーション機能やアルバム機能の実装を行った。
\end{hissu}

\subsubsection{第三開発}
\begin{hissu}
第三開発では、第二開発で立てた計画や設計を基に実装作業、第二開発で頂いた評価を参考に
マニュアルや話題と豆知識の修正作業を行った。実装作業では、コミュニケーション機能を中心に、
写真のやりとりやアルバムの表示、話題と豆知識の表示を行った。マニュアルは、利用者の使いやすさを
考え、ブラウザ上で表示できるようにHTMLとCSSを用いたコードに変更した。話題と豆知識は、
第二開発で頂いた評価を参考に提供する情報を再検討し、情報の内容をより詳細に決めた。
アプリの内容やデザイン、操作手順などに関する評価を得るために、開発に協力して頂いた
成本医師や医療関係者の永山さんに伺った。また、最終成果発表会では、実装したアプリを実際に
使ってもらい、アプリのアイディアやデザイン、使いやすさなどについての評価を頂いた。

\end{hissu}

\subsection{今後の課題と展望}
\begin{hissu}
本グループが開発した、iPadのクローズドSNSアプリ「NiCoRe」は、将来的にAppStoreに申請し
公開することを想定している。しかし、まだ実装仕切れていない機能がある。また、豆知識の
トピックについて、医療関係者からの情報提供や内容の評価を受ける必要があるため、リリースには時間が
かかる見通しである。そのため、今後は完成した認知症マニュアルの部分だけのリリースを目指し、実装を
行う。そして、段階的にNiCoReのリリースを行っていき、最終的には、作成したアプリを
実際の高齢者とその家族に遠隔で使用してもらい、アプリの評価を集めていくことが今後の課題である。


\end{hissu}





















\section{教育系}
\subsection{はじめに}
\begin{hissu}
私たち教育系グループはこの1年間を通して、プログラミング教育という視点でアプリの開発を
行った。自分たちのアプリでプログラミングを理解して楽しんでもらうという目的で
あった。しかし私たちでグループではアプリ案が幾度となく変更があったため前期と後期では
アプリのテーマやコンセプトがそれぞれ異なってしまった。
\end{hissu}

\subsection{前期の背景}
\begin{hissu}
現在、日本の中学校では2012年から中学校の技術家庭科でプログラミング教育が
必修項目となっている。しかし、今の中学校のプログラミング教育ではソースコードを
打ってプログラミングをするということを行っておらず、プログラミングの内容を
深く取り上げていない。
\end{hissu}

\subsection{課題の設定と到達目標}
\begin{hissu}
私たちは中学で学んだプログラミングと実際のプログラミングの間のプロセスを支援するゲームアプリを
開発することに決めた。自分たちのアプリでプログラミングを理解して楽しんでもらうという目的で
あった。
\end{hissu}

\subsection{課題解決のプロセスとその結果}

\subsubsection{前期の活動}
\begin{hissu}
テーマが決まった後、アプリの設計を行った。しかし、要件定義を固めずにアプリの設計を行ったため、
一貫性のないアプリ設計になってしまった。そのため、要件定義を1からやり直すことになった。
要件定義をやり直すことは、1度で終わらず何度も行った。その結果、中学校でプログラミングを
学んだ人、興味を持った人を対象にしたゲームアプリというテーマを決めた。
\end{hissu}
\subsubsection{開発アプリ}
\begin{hissu}
そして中間発表会までにアプリのプロトタイプを完成させた。このアプリは中学生でプログラミングを習った人、興味を持った人を対象としたソースコードの組み方を学ぶゲームアプリである。
このゲームには自機と敵機があり、ユーザはマス目上のステージにある自機をソースコードを
組むことによって動かし、敵機を倒すことでゲームがクリアとなる。
\end{hissu}
\subsubsection{中間発表会}
\begin{hissu}
中間発表では、私たちが考えたアプリをポスターにまとめ、ポスターセッションを行った。
教員や他学生からの評価シートには「最終的なゴールは?」、「まだ内容が決まっていないので
評価不能」、「既存のもとの比較がない」などの意見をいただき、
もう1度、要件定義を見直しアプリの設計をやり直す必要があることに気付かされた。
\end{hissu}
\subsection{後期の背景}
\begin{hissu}
公立はこだて未来大(以下、未来大とする)には、学部1年生が後期に必修科目として履修する「プログラミング基礎」
という講義がある。この講義は、1年生が前期に履修した「情報表現入門」で学んだことをもとに、
C言語について学び、プログラミング(概念、考え方)への理解を深めることを目的としている。
講義内容はプログラミング基礎概念である、変数、配列、条件分岐、繰り返し、関数、文字列、
構造体などについて学ぶ内容となっている。また、講義だけでなく、演習として
課題(プログラム)に取り組むことになっており、毎回の講義で課題を提出することになっている。
しかし、例年「プログラミング基礎」の落第者が多くいる。「プログラミング基礎」は
必修科目であるため、卒業するためには必ず単位を取らなければいけない講義である。
\end{hissu}

\subsection{課題の設定と到達目標}
\begin{hissu}
C言語を学んでいる大学生を対象としたC言語のWebアプリの学習教材を
開発することをテーマに決めた。
\end{hissu}

\subsection{課題解決のプロセスとその結果}
\subsubsection{後期の活動}
\begin{hissu}
現在、C言語を学んでいる未来大の学生とその講義を担当している
TA(Teaching Assistant)にヒアリング調査を行った。その結果、C言語の配列などの概念や
アルゴリズムが分かっていないことが分かった。また、現在使われている講義資料を
確認してみたところ、ほとんどが文字で構成されており、専門用語が多くあり、
C言語を初めて学ぶ学生にとっては分かりにくい講義資料だと思った。そこで、私たちは
これらの課題を解決するためアニメーションで、概念やアルゴリズムを教えるWebアプリを
開発することを決めた。
\end{hissu}

\subsubsection{開発アプリ}
\begin{hissu}
最終成果発表に向けて開発したものは「C-mation」というC言語のプログラミング
学習支援ツールである。対象ユーザは未来大1年生である。ユーザにはこのWebアプリを
用いてC言語のプログラミングを学んでもらおうと考えた。このアプリではアニメーションを
用いて、概念の説明、例題の解説、確認問題の出題を行う。このように段階を踏んで教えることで、
C言語を理解してもらうことが目的である。最終成果発表会までに配列の部分の開発を行った。

\end{hissu}

\subsubsection{最終成果発表会}
\begin{hissu}
最終発表では、中間発表と同様に私たちが考えた提案をポスターにまとめ、ポスターセッションを
行った。教員や他学生からの評価シートには「イメージがわくので、分かり易かったです」
などの意見をいただき前期と比べて評価は高かった。一方で、「実際に1年生に使ってほしかった」、
「他の班と比べてiOSアプリを作らなかったメリットを知りたい」などの意見をいただき、
今後の改善点を見つけることができた。

\end{hissu}

\subsection{今後の課題と展望}
\begin{hissu}
今後の展望としては、私たちが作成した提案物と酷似したソフトウェアが本学の2年次の科目である
「アルゴリズムとデータ構造」の教科書の付属CDにあったため、そのソフトウェアとの差別化を
図っていきたい。また、ユーザに評価していただくために本学の1年生やメタ学習センターに
実際に使っていただきたいと思う。

\end{hissu}

\end{document}
